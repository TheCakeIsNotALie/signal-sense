% !TeX spellcheck = fr_FR
\thispagestyle{noheader}
\chapter*{Résumé} % No (numbered) toc entry with *

\tikz[remember picture,overlay] \node[shift={(4.165cm,-1.955cm)}]
	at (current page.north west)
	{\includegraphics[height=1.29cm]{template/images/title/hepia_logo}};
	\tikz[remember picture,overlay] \node[shift={(-4.238cm,-1.97cm)}]
	at (current page.north east)
	{\includegraphics[height=1.29cm]{template/images/title/hes-so_geneve_logo}};

\addcontentsline{toc}{chapter}{Résumé} % Adding toc entry
\thispagestyle{noheader}

\begin{spacing}{0.956}
\vspace{0.5cm}

La lumière Cherenkov est un phénomène physique similaire à un boom supersonique dans le domaine électromagnétique.
Lorsque des particules chargées entrent dans notre atmosphère à la vitesse de la lumière, 
elles entament une réaction en chaîne qui produit une pluie de lumière atmosphérique.
L’observation de ces pluies nous permet de mieux comprendre les mécanismes de la physique des particules. 
Dans cette optique, l’UNIGE travaille sur la mission pionnière du télescope spatial Terzina. 
Ce télescope a pour but de détecter et d’étudier les neutrinos de manière indirecte en observant le limbe de la Terre 
et en y détectant des pluies de lumière atmosphérique. Lors de sa conception, il s'est avéré 
que le télescope recueillera plus de données qu'il sera capable de transmettre au sol.
% En effet, Terzina n’aura accès qu'à 40 Gbit de transmission par jour.
Le but initial de ce projet consistait à créer un modèle de réseau de neurones embarquable à bord du satellite,
afin de réduire l'utilisation de la bande passante, pour cela le réseau neuronal effectuera un analyse des données des capteurs
afin de détecter la quantité de photons provenant d'une pluie atmosphérique et aider à cibler des événements plus impactants.  
Suite à des délais de production de certains composants essentiels au fonctionnement du réseau neuronal à bord du satellite,
le projet a été adapté pour des télescopes au sol. Ceux-ci rencontrent des problèmes similaires de haute utilisation 
de stockage, de resources de calcul à cause de la quantité astronomique de données recueillies.
Un réseau de neurones similaire pourrait donc leur être utile.
Lors de ce travail, j'ai commencé par étudier le phénomène physique et le fonctionnement des capteurs. 
J'ai ensuite pris en main prendre en main les différents simulateurs de données. 
Avec ces données, j'ai testé différentes architectures de réseau neuronaux pour trouver un modèle performant.

\vfill
\begin{center}
	{\includegraphics[width=0.4\linewidth]{MagicCalibration.jpg}}\\*
\vfill
%% CONTENT ENDS HERE

{
%%%%%%%%%%%%%%%%%%%%%%%%%%%%%%%%%%%%%%%%%%%%%%%%%%%%%%%%%%%%%%%%%%%%%%%%%%%%%%%%
%%%%%%%%%%%%%%%%%%%%%%%%%% DO NOT MODIFY THE TABLE BELOW %%%%%%%%%%%%%%%%%%%%%%%
%%%%%%%%%%%%%%%%%%%%%%%%%%%%%%%%%%%%%%%%%%%%%%%%%%%%%%%%%%%%%%%%%%%%%%%%%%%%%%%%
	\begin{tabular*}{16cm}{p{7.59cm} p{7.58cm}}
		\small Candidat-e:					&	\small Professeur-e(s) responsable(s):\\*[10pt]
		\small\textbf{\textsc{\Author}}		&	\small\textbf{\textsc{\Professor}}\\*[10pt]
		\footnotesize  Filière d’études : ISC	&	\footnotesize  \textbf{En collaboration avec:} UNIGE\\*[10pt]
		\footnotesize  {} & \footnotesize  Travail de bachelor soumis à une convention de stage en entreprise: \Convention\\*[20pt]
		\footnotesize  {} & \footnotesize  Travail soumis à un contrat de confidentialité: \Confidentiel\\*[10pt]
	\end{tabular*}\\*[1.9cm]
}
\end{center}
\end{spacing}
