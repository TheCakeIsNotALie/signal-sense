% !TeX spellcheck = fr_FR
% \chapter*{Bibliographie}

% \noindent\textit{Sites Web consultés – Code repris d’ailleurs – Notices techniques – Articles de presse – Ouvrage imprimés – Ouvrages électroniques – Chapitre dans un ouvrage imprimé – Rapports imprimés – Travaux universitaires – Articles de revues imprimés – Articles de périodiques électroniques – Communication dans un congrès. Pour chacun de ces types de document, les mise en forme sont dans le document « Méthode de citation et de rédaction d’une bibliographie ».}\\

% \textit{Afin de gagner du temps, pensez à utiliser le logiciel de gestion bibliographique Zotero (et/ou BibTeX si vous utilisez LaTeX) pour la mise en forme et l’édition automatique de vos références à la norme ISO690.}
\nocite{*}
\addcontentsline{toc}{chapter}{Bibliographie} % Adding toc entry
\printbibliography