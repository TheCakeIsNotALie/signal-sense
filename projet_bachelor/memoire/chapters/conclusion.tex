% !TeX spellcheck = fr_FR
\chapter*{Conclusion}


Choses qui auraient pû être mieux faites :
Tester des réseaux de neurones sur des datasets fixes (fait sur la fin mais le faire depuis le début)

Améliorations possible :
Minimiser le bruit électronique en incluant les 6 voisins dans l'estimation de \gls{pe}



\addcontentsline{toc}{chapter}{Conclusion} % Adding toc entry

Étant passionné par l'astronautique et intéressé 
par tous les sujets adjacents tels que l'astrophysique ou même l'aéronautique, 
j'ai été ravi d'être choisi pour participer à un projet concret touchant à ces mêmes domaines.

L'objectif de ce projet de semestre consistait à préparer le développement d'un modèle de réseau neuronal
capable d'aider au tri d'évènements capturés par la caméra de Terzina. 
Il s'est avéré au milieu du projet que les capacités techniques à bord du satellite ne seraient plus capable 
de produire des données suffisantes pour le réseau neuronal envisagé, mais les résultats préliminaires du projet ont permis 
à l'équipe de l'\gls{unige} de voir un potentiel dans la réutilisation d'un tel modèle dans des télescopes \gls{iact} terrestres.

La première étape de ce travail a été de comprendre le phénomène physique
de la lumière Cherenkov. Cette étape de découverte m'a grandement plu car pendant mes recherches sur le sujet,
j'ai redécouvert des sujets dont j'avais déjà entendus parler et ai pu en approfondir mes connaissances.

Dans un deuxième temps, j'ai dû comprendre et faire fonctionner des projets existants pour générer les données 
sur lesquelles baser mes essais de réseau de neurones.
Lors de mes premiers essais, j'ai rapidement rencontré les différents cas bloquants vus dans mes cours de machine learning,
notamment les problèmes liés aux format et à la composition des données d'entrée.

Ces différentes étapes m'ont permis d'établir une liste des tâches et de fonctionnalités nécessaires au projet final.
Mon projet de bachelor sera donc d'établir des métriques de performance et une série de tests pour pouvoir comparer efficacement 
les prochains modèles qui seront étudiés lors de la suite de ce projet afin de trouver un ou plusieurs modèle adapté aux attentes de l'équipe de l'UNIGE.
