% !TeX spellcheck = fr_FR
\chapter*{Conclusion}

\addcontentsline{toc}{chapter}{Conclusion} % Adding toc entry

Étant passionné par l'astronautique et intéressé par tous les sujets adjacents tels que l'astrophysique ou même l'aéronautique, 
j'ai été ravi d'être choisi pour participer à un projet concret touchant à ces mêmes domaines.

Pour rappel, l'objectif de ce projet de bachelor consistait à trouver un modèle de réseau neuronal capable d'interpréter 
le contenu de signaux produits par des télescopes afin d'identifier les possibles photons provenant de pluies atmosphériques Cherenkov.
Le modèle trouvé aurait servi à deux usages : le premier à réduire le stockage et le traitement des données en participant 
au tri d'événements scientifiquement intéressants, le deuxième à possiblement améliorer les performances d'analyse de plus haut niveau 
en filtrant le bruit contenu dans les signaux. 

Pour réaliser ce projet, j'ai d'abord dû appréhender le phénomène physique de la radiation Cherenkov et ses implications pour le domaine de l'astrophysique.
Il m'a fallu aussi comprendre le fonctionnement des différents télescopes existants et futurs afin de discerner 
ce qui était attendu de ce projet. J'y ai appliqué les bases en machine learning que j'ai appris lors de mon cursus, et ai 
réussi à éviter plusieurs pièges du domaine. J'ai aussi dû prendre en main plusieurs outils simulant des données scientifiques et les adapter
pour entraîner des modèles de machine learning. Malgré les diverses techniques de machine learning testées,
aucun des modèles n'a présenté des performances permettant à celui-ci d'être intégré au sein d'autres projets comme imaginé.

Cependant, la détection de \gls{pe} d'un signal n'est pas encore a considérer comme impossible. 
Il reste encore des techniques de machine learning plus complexes a investiguer. 
De plus et comme évoqué au chapitre précédent, si mon soupçon que les données ne contiennent pas assez de détails s'avérait exact, il est envisageable d'observer
les données d'une manière différente pour minimiser les effets de bruits. Par exemple, en observant des groupes de pixels voisins. 
Il se pourrait aussi qu'il est impossible de discerner la présence de petites quantités de \gls{pe} dans des pluies de très basse énergie.

Dans l'ensemble, je suis satisfait du travail que j'ai pû fournir au cours de ces derniers mois.
Grâce à cette expérience, j'ai pu améliorer mes connaissances en machine learning et en recherches scientifiques.
Dans un monde se tournant de plus en plus vers les intelligences artificielles et le machine learning,
il est certain que les apprentissages et les conclusions que j'ai acquis au cours de ce travail me seront d'une 
grande utilité durant toute ma carrière professionnelle.