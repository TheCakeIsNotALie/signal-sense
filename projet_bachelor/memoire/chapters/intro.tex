% !TeX spellcheck = fr_FR
\chapter*{Introduction}
\addcontentsline{toc}{chapter}{Introduction} % Adding toc entry

Dans le cadre de mon travail de bachelor, j'ai été choisi pour travailler avec le groupe "High-Energy Multi-Messenger" de l'UNIGE.
Ce groupe effectue des recherches expérimentales sur les particules provenant d'astres lointains à l'aide de satellites et de télescopes.
Leurs activités se concentrent majoritairement sur les rayons gamma, les rayons cosmiques et les neutrinos.
J'ai été mandaté pour les aider à traiter des données concernant les rayons gamma et cosmiques à travers le phénomène physique de la radiation Cherenkov. 

Ce phénomène se produit lorsqu'une particule chargée se déplace à une vitesse supérieure à celle de la lumière dans un médium diélectrique transparent. 
L'exemple le plus notable de ce phénomène est la lumière bleutée produite dans l'eau autour d'un réacteur nucléaire. 
Cette lumière est due aux rayons gamma produits par la réaction nucléaire. Ces rayons, à cause des immenses forces électromagnétiques, se déplacent momentanément plus 
rapidement que la vitesse de la lumière dans l'eau.
On peut comparer ce phénomène à un boom supersonique mais pour de la lumière.

Les chercheurs de l'UNIGE s'intéressent cependant moins aux rayons gamma produits par des installations humaines qu'aux rayons et particules cosmiques provenant de l'espace.
Ces rayons et particules extraterrestres produisent aussi une radiation de Cherenkov mais dans le médium de notre atmosphère.
Lorsque le phénomène Cherenkov débute dans l'atmosphère, on appelle cela la "lumière Cherenkov".
Ce rayonnement va créer un cône de particules chargées qui sera détecté au sol à l'aide de télescopes. 

Aujourd'hui, ce phénomène est déjà étudié par de nombreux télescopes terrestres mais son observation est limitée par
divers bruits. Pour pallier à ceux-ci, il est prévu d'étudier ce phénomène depuis un télescope en orbite afin d'éviter ces bruits au maximum.
Cependant, le satellite envisagé possède des limitations techniques notamment sur la quantité d'informations qu'il peut envoyer ou recevoir.
Mon implication dans ce projet est de présenter un modèle de machine learning qui, en effectuant une étape de pré-traitement, aidera à cibler les
évènements détectés et, possiblement, réduire la quantité de données à transmettre.

La première partie de ce rapport portera sur une explication du phénomène physique et de la façon dont il est détecté physiquement.
Une deuxième partie sur les observateurs existants et l'intérêt d'observer ce phénomène depuis l'espace.
Une troisième partie expliquant les technologies, outils et méthodes utilisées au cours de ce projet.
Et, en dernier, les résultats obtenus ainsi que des comptes rend.