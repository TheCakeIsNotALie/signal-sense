% Acronym definitions
\newacronym{em}{EM}{Électromagnétique}
\newacronym{ctao}{CTAO}{Cherenkov Telescope Array Observatory}
\newacronym{iact}{IACT}{Imagin Athmospheric (or Air) Cherenkov Telescope}
\newacronym{esa}{ESA}{Euopean Space Agency}
\newacronym{le}{LE}{Low Energy}
\newacronym{he}{HE}{High Energy}
\newacronym{vhe}{VHE}{Very High Energy}
\newacronym{uhe}{UHE}{Ultra High Energy}
\newacronym{ehe}{EHE}{Extremely High Energy}
\newacronym{unige}{UNIGE}{Université de Genève}
\newacronym{infn}{INFN}{Istituto Nazionale di Fisica Nucleare}
\newacronym{leo}{LEO}{Low Earth Orbit}
\newacronym{sipm}{SiPM}{Silicon Photomultiplier}
\newacronym{nasa}{NASA}{National Aeronautics and Space Administration}
\newacronym{fpga}{FPGA}{Field Programmable Gate Arrays}
\newacronym{sst}{SST}{Small-Sized Telescope}
\newacronym{mst}{MST}{Medium-Sized Telescope}
\newacronym{lst}{LST}{Large-Sized Telescope}
\newacronym{asic}{ASIC}{Application Specific Integrated Circuits}
\newacronym{cnn}{CNN}{Convolutional Neural Network}
\newacronym{pe}{pe}{Photon-Électron}
\newacronym{hepia}{HEPIA}{Haute École du paysage, d'ingénierie et d'architecture de Genève}
\newacronym{cern}{CERN}{Conseil Européen pour la Recherche Nucléaire}
